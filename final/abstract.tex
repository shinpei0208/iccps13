\begin{abstract}
 Cyber-physical systems (CPS) aim to monitor and control complex real-world
 phenomena where the computational cost and real-time constraints could
 be a major challenge.
 Many-core hardware accelerators such as graphics processing units (GPUs)
 promise to enhancing computation, leveraging the data parallelism often
 found in real-world scenarios of CPS, but performance is limited by the
 overhead of the data transfer between the host and the device memory.
 For example, plasma control in the HBT-EP Tokamak device at
 Columbia University~\cite{Maurer_PPCF11,Rath_FED12} must execute the
 control algorithm in a few microseconds, but may take tens of
 microseconds to copy the data set between the host and the device
 memory.
 This paper presents a zero-copy I/O processing scheme that maps
 the I/O address space of the system to the virtual address space of the
 compute device, allowing sensors and actuators to transfer data to and
 from the compute device directly. 
 Experiments using the plasma control system show a 33\% reduction in
 computational cost, and microbenchmarks with more generic matrix 
 operations show a 34\% reduction, while in both cases, effective data 
 throughput remains at least as good as the current best  performers.
\end{abstract}
