\section{System Implementation}
\label{sec:implementation}

GPU programs often execute in \textit{virtual address spaces}.
In CUDA programming, for instance, a device pointer acquired by the
memory allocation API functions, such as {\tt cuMemAlloc()} and {\tt
cuMemAllocHost()}, represents a virtual address.
The relevant data-copy API functions, such as {\tt cuMemcpyHtoD()} and {\tt
cuMemcpyDtoH()}, also use this pointer to the virtual address rather
than a physical address.
As long as the programs remain within the GPU or the CPU, this is a
suitable addressing model.
However, CPS applications often require I/O devices for sensing and
actuation.
The current software stack and the API design of GPU programming force
these applications to use the host memory as an intermediate buffer to
bridge between the I/O device and the GPU.
No prior system implementation has allowed data to move directly between
them. 

In this section, we present our system implementation of the {\dm} and
{\dmh} schemes using Gdev~\cite{Kato_ATC12}, an open-source facility of
the device driver and the runtime library for NVIDIA GPUs.
A key limitation is that the data access of I/O devices is limited to
the I/O address space.
However, because the GPU is typically connected to the system via the
PCIe bus, we can map the virtual address space of the device memory to
the reserved I/O address space of the PCI bus, as mentioned in
Section~\ref{sec:dm}.
In this way, I/O devices can directly access data in the device memory.
Specifically, their device drivers can configure their hardware direct
memory access (DMA) engines to target the physical address associated
with the PCI-mapped space of the device memory.

Our system implementation adds the three functions presented in
Section~\ref{sec:dm} to Gdev.
While Gdev is a Linux kernel module for first-class GPU resource
management, it also provides an implementation of the CUDA Driver
API~\cite{CUDA} for user programming.
The three functions are wrapped by the Gdev-original extended API
functions, {\tt cuMemMap()}, {\tt cuMemUnmap()}, and {\tt
cuMemGetPhysAddr()}, which are compatible to the form of the CUDA Driver
API.
We now provide the details of these functions:

\begin{itemize} \itemsep1pt
 \item \textbf{Mapping Memory:}
       We assign the second PCIe BAR region, \textit{i.e.}, BAR1 (among
       the five of those supported by hardware as a window of the device
       memory), because this is the largest region, often equal to 128MB,
       for NVIDIA GPUs.
       Since these GPUs provide an MMIO register to point the PCIe BAR
       to any virtual address of the device 
       memory, we create special virtual address space dedicated to the
       PCIe BAR when the system is loaded.
       When the user context requests mapping of some device memory
       object, we first find the physical pages allocated to this
       memory object.
       We next set up the page table of the GPU so that these
       physical pages are referenced by the virtual address space
       dedicated to the PCIe BAR.
       Now, these physical pages are referenced by two virtual address
       spaces: one dedicated to the PCIe BAR and the other associated
       with the user context.
       As a result, the data written to the PCIe BAR can be referenced
       by the user context and the mapping is done.
       If the user context furhter needs to map the same memory to the
       host virtual address, we can use a Linux kernel primitive such as
       {\tt ioremap()}.
 \item \textbf{Unmapping Memory:}
       To unmap the allocated space of the device memory from the PCIe
       BAR, we simply delete the corresponding record of the page table.
       If host memory is also mapped, we also call a Linux kernel
       primitive such as {\tt iounmap()}.
 \item \textbf{Getting Physical Address:}
       The operating system kernel usually provides a function to return
       the physical addresses of PCIe BARs.
       In the Linux kernel, we can use {\tt pci\_resource\_start()} for
       this purpose.
\end{itemize}
