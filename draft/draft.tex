% THIS IS SIGPROC-SP.TEX - VERSION 3.1
% WORKS WITH V3.2SP OF ACM_PROC_ARTICLE-SP.CLS
% APRIL 2009
%
% It is an example file showing how to use the 'acm_proc_article-sp.cls' V3.2SP
% LaTeX2e document class file for Conference Proceedings submissions.
% ----------------------------------------------------------------------------------------------------------------
% This .tex file (and associated .cls V3.2SP) *DOES NOT* produce:
%       1) The Permission Statement
%       2) The Conference (location) Info information
%       3) The Copyright Line with ACM data
%       4) Page numbering
% ---------------------------------------------------------------------------------------------------------------
% It is an example which *does* use the .bib file (from which the .bbl file
% is produced).
% REMEMBER HOWEVER: After having produced the .bbl file,
% and prior to final submission,
% you need to 'insert'  your .bbl file into your source .tex file so as to provide
% ONE 'self-contained' source file.
%
% Questions regarding SIGS should be sent to
% Adrienne Griscti ---> griscti@acm.org
%
% Questions/suggestions regarding the guidelines, .tex and .cls files, etc. to
% Gerald Murray ---> murray@hq.acm.org
%
% For tracking purposes - this is V3.1SP - APRIL 2009

\documentclass{sig-alternate-ipsn13}

%-----------------------------------------
% need for camera-ready
\pagestyle{empty}

%----------------------------------------

\begin{document}

\title{
Zero-Copy I/O Processing for Real-Time GPU Applications
}
%
% You need the command \numberofauthors to handle the 'placement
% and alignment' of the authors beneath the title.
%
% For aesthetic reasons, we recommend 'three authors at a time'
% i.e. three 'name/affiliation blocks' be placed beneath the title.
%
% NOTE: You are NOT restricted in how many 'rows' of
% "name/affiliations" may appear. We just ask that you restrict
% the number of 'columns' to three.
%
% Because of the available 'opening page real-estate'
% we ask you to refrain from putting more than six authors
% (two rows with three columns) beneath the article title.
% More than six makes the first-page appear very cluttered indeed.
%
% Use the \alignauthor commands to handle the names
% and affiliations for an 'aesthetic maximum' of six authors.
% Add names, affiliations, addresses for
% the seventh etc. author(s) as the argument for the
% \additionalauthors command.
% These 'additional authors' will be output/set for you
% without further effort on your part as the last section in
% the body of your article BEFORE References or any Appendices.

\numberofauthors{2} %  in this sample file, there are a *total*
% of EIGHT authors. SIX appear on the 'first-page' (for formatting
% reasons) and the remaining two appear in the \additionalauthors section.
%
\author{
% You can go ahead and credit any number of authors here,
% e.g. one 'row of three' or two rows (consisting of one row of three
% and a second row of one, two or three).
%
% The command \alignauthor (no curly braces needed) should
% precede each author name, affiliation/snail-mail address and
% e-mail address. Additionally, tag each line of
% affiliation/address with \affaddr, and tag the
% e-mail address with \email.
%
\alignauthor Shinpei Kato\\
       \affaddr{Department of Information Engineering}\\
       \affaddr{Nagoya University}
\and
\alignauthor Jason Aumiller and Scott Brandt\\
       \affaddr{Department of Computer Science}\\
       \affaddr{University of California, Santa Cruz}
\and
\alignauthor Nikolaus Rath\\
       \affaddr{Department of Applied Physics and Applied Mathematics}\\
       \affaddr{Columbia University}
}
% There's nothing stopping you putting the seventh, eighth, etc.
% author on the opening page (as the 'third row') but we ask,
% for aesthetic reasons that you place these 'additional authors'
% in the \additional authors block, viz.
%\additionalauthors{Additional authors: John Smith (The Th{\o}rv{\"a}ld Group,
%email: {\texttt{jsmith@affiliation.org}}) and Julius P.~Kumquat
%(The Kumquat Consortium, email: {\texttt{jpkumquat@consortium.net}}).}
%\date{30 July 1999}
% Just remember to make sure that the TOTAL number of authors
% is the number that will appear on the first page PLUS the
% number that will appear in the \additionalauthors section.

\maketitle

%-----------------------------------------
% need for camera-ready
\thispagestyle{empty}

\begin{abstract}
 Cyber-physical systems (CPS) often control complex physical
 phenomenon. 
 The computational workload of control algorithms, hence, is becoming a
 core challenge of CPS due to their real-time constraints.
 By nature, CPS control algorithms exhibit a high degree of data
 parallelism, which can be offloaded to parallel compute devices,
 such as graphics processing units (GPUs).
 Yet another problem is introduced by the communication overhead between
 the host processor and the compute device.
 As a matter of fact, plasma fusion requires a sampling period of a few
 microseconds, while today's systems may take several ten microseconds
 to copy data between the host and the device memory at scale of the
 required data size.
 In this paper, we present a zero-copy I/O processing scheme that
 enables sensor and actuator devices to directly communicate with
 compute devices without accessing the host processor.
 This scheme maps the I/O address space to the device memory to remove
 data-copy operations with respect to the host memory.
 The experimental results from Columbia University's ``Tokamak'' fusion
 control system demonstrate that a sampling period of plasma fusion can
 be reduced by 33\% under the zero-copy I/O scheme.
 The microbenchmarking results also show that GPU-accelerated
 tasks can be completed in 34\% less time than current methods, while
 effective data throughput is at least as good as the best performers of
 current methods.
\end{abstract}

\keywords{GPGPU, Zero-Copy I/O, Plasma Fusion}

\section{Introduction}
\label{sec:introduction}

\cite{Kato_ATC11}.\cite{Kato_RTAS11}.

\bibliographystyle{abbrv}
\bibliography{references}

\end{document}
