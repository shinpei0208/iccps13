\begin{abstract}
 Cyber-physical systems (CPS) aim to control complex real-world
 phenomenon.
 Due to real-time constraints, however, the computational cost of
 control algorithms is becoming a major issue of CPS.
 Parallel computing of the control algorithms using state-of-the-art
 compute devices, such as graphics processing units (GPUs), is a
 promissing approach to reduce this computational cost, given that
 real-world phenomenon by nature compose data parallelism, yet another
 problem is introduced by the overhead of data transfer between the host
 processor and the compute device.
 As a matter of fact, plasma control requires an order of a few
 microseconds for the control rate, while today's systems may take an
 order of ten microseconds to copy the corresponding problem size of
 data between the host processor and the compute device, which is
 unacceptable lantecy.
 In this paper, we propose a zero-copy I/O processing scheme that
 enables sensor and actuator devices to directly transfer data to and
 from the compute device.
 The basic idea behind this scheme is to map I/O address space,
 accessible to sensor and actuator devices, to virtual memory space of
 the compute device.
 The results of experiments using the real-world plasma control system
 demonstrates that the computational cost of the plasma control
 algorithm is reduced by 33\% under our new scheme.
 We further provide the results of microbenchmarking to show that more
 generic matrix computations are completed in 34\% less time than
 current methods, using our new scheme, while effective data throughput
 remains at least as good as the current best performers.
\end{abstract}
