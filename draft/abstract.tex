\begin{abstract}
 Cyber-physical systems (CPS) often control complex physical
 phenomenon. 
 The computational workload of control algorithms, hence, is becoming a
 core challenge of CPS due to their real-time constraints.
 By nature, control algorithms of CPS exhibit a high degree of data
 parallelism, which can be offloaded to parallel compute devices,
 such as graphics processing units (GPUs).
 Yet another problem is introduced by the communication between the host
 processor and the compute device.
 As a matter of fact, plasma control requires an order of a few
 microseconds for the sampling period, while today's systems may take
 several ten microseconds to copy data between the host and the device
 memory at scale of the required data size.
 In this paper, we present a zero-copy I/O processing scheme that
 enables sensor and actuator devices to directly transfer data to and
 from compute devices without using the host processor.
 The basic idea behind this scheme is to map the I/O address space onto
 the device memory, removing data-copy operations upon the host memory.
 The experimental results from the real-world plasma control
 system demonstrate that a sampling period of plasma control can be
 reduced by 33\% under the zero-copy I/O scheme.
 The microbenchmarking results also show that GPU-accelerated matrix
 computations can be completed in 34\% less time than current methods,
 while effective data throughput is at least as good as the current best
 performers.
\end{abstract}
