\begin{abstract}
 Cyber-physical systems (CPS) aim to control complex real-world
 phenomenon where the computational cost and real-time constraints could
 be a major challenge.
 Parallel computing technology using compute devices such as graphics
 processing units (GPUs) promises to enhancing computation, leveraging
 the data parallelism often found in real-world scenarios, but
 performance is limited by the overhead of the data transfer between the
 host and the device memory.
 For example, plasma control in the HBT-EP ``Tokamak'' device at Columbia
 University~\cite{Maurer_PPCF11,Rath_FED12} must run the control
 algorithm in a few microseconds, but may take tens of microseconds to
 copy the data set between the host and the device memory.
 We present a new zero-copy I/O processing scheme that maps
 the I/O address space of the system to the virtual address space of the
 compute device, allowing sensor and actuator devices to transfer data
 to and from the compute device directly. 
 Experiments with the plasma control system show a 33\% reduction in
 computational cost, and microbenchmarks with more generic matrix 
 operations show a 34\% reduction, while in both cases, effective data 
 throughput remains at least as good as the current best  performers.
\end{abstract}
